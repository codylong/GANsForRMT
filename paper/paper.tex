% Document class
\documentclass[12pt,a4paper]{article}
\usepackage[left=2.3cm, right=2.3cm, top=2.54cm]{geometry}

% Package imports
\usepackage{amsmath}
\usepackage{amsfonts}
\usepackage{graphicx}
\usepackage[font=small,labelfont=bf]{caption}
\usepackage{epstopdf}
\usepackage{authblk}
\usepackage{subcaption}
\usepackage{multirow}
\usepackage{tikz}
\usepackage{hyperref}
\usetikzlibrary{positioning}
\usetikzlibrary{backgrounds}

\newcommand{\jh}[1]{\noindent{\color{blue}\textbf{JH:} #1}}
\newcommand{\ra}[1]{\noindent{\color{orange}\textbf{RA:} #1}}
\newcommand{\jc}[1]{\noindent{\color{red}\textbf{JC:} #1}}

% Commands
\def\GLnZ{$GL(n,\mathbb{Z})$ }
\def\GLfZ{$GL(4,\mathbb{Z})$ }
\def\h11{h^{1,1}}
\def\cpp{C\texttt{++} }
\def\D491{\Delta^{\circ}_{491}}

\newcommand{\gappeq}{\mathrel{\rlap {\raise.5ex\hbox{$>$}}
{\lower.5ex\hbox{$\sim$}}}}
\newcommand{\lappeq}{\mathrel{\rlap{\raise.5ex\hbox{$<$}}
{\lower.5ex\hbox{$\sim$}}}}



\title{\textbf{Learning Random Matrix Approximations of String Theory: K\" ahler Metrics}}
\author[]{James Halverson}
\author[]{Cody Long}

\date{}

\affil[]{Department of Physics, Northeastern University\\Boston, MA 02115, USA}

\begin{document}
\maketitle


\vspace{1cm}

\begin{abstract}
We do lots of interesting things.
 \end{abstract}

\clearpage


\tableofcontents


%\section{Introduction}

\section{Introduction}

\section{Methods}
\subsection{Optimization of Bergman Metrics}
\subsubsection{A brief review of Bergman metrics}
In the same way that many functions can be well-approximated by a Taylor series,
 one might hope to find a universal class of metrics that converge to the metrics of interest.
For K\"ahler metrics, the class of Bergman metrics serve this purpose. It was shown by
(Tian-Yau-Zelditch) that a Bergman metric can provide an arbitrarily good approximation to
a given K\"ahler metric, in a fixed K\"ahler class. To review, let us consider a compact K\"ahler
manifold $M$ with an effective line bundle $L$, such that $L^k$ is ample for some $k > 0$. 
We consider a basis of sections of $L^k$ of the form $s^\alpha (z^i)$, where the $z^i$ are 
projective coordinates on $M$. The K\"ahler potential for the Bergman metric takes the form
\begin{equation}
k = \mathrm{log}(s_\alpha P_{\alpha \bar{\beta}} \bar{s}_{\bar{\beta}})\, , 
\end{equation}
where the matrix $P$ is positive-definite, and therefore can be written as $P = A^\dagger A$. 
\subsection{New Ensembles from Generative Adversarial Networks}

\section{K\" ahler metrics at fixed Picard Number}

\subsection{$h^{11} = 10$}
\subsection{$h^{11} = 16$}

\section{Interpolation and Extrapolation in Picard Number}

\section{Discussion}

\clearpage 


\end{document}